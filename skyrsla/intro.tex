\section{Inngangur}
Við ætlum að búa til bíl, alveg eins og sá sem við bjuggum til á fyrstu önn, og setja arm með svona kló á hann sem verður stjórnað af gyroscope. Verða þá arm-mótorar tengdir Arduino í stað RobotC heilans og notum við gyroscope til að segja til um hversu mikill halli á að vera á arminum og hvar hann á að vera. Ofan á þetta ætlum við að nota Arduino myndavél og reyna að senda live video feed í tölvu, til þess að við getum séð hvað er í gangi á hverri stundu.
Þá verðum við með bíl sem stjórnað er af fjarstýringu og Arduino gyroscope og myndavél sem sendir live video feed í tölvu sem við getum horft á á sama tíma.
Í heild sinni verður bíllinn svipaður og sá sem við gerðum á fyrstu önn, en það verður það sem við ætlum að reyna að vinna með.
Við fáum hluti í bíl og Arduino tölvu frá kennaranum og ætlum að byrja verkefnið á að byggja bílinn, koma svo gyroscope fyrir og á endanum ná live video-i í tölvuna. Við ætlum að nota fjarstýringu sem fylgir RobotC bílunum og Arduino Mega borð.
Þegar við erum komnir með controls á hreint ætlum við að reyna að fá live feed myndband, sem gæti verið flóknara, en ekki ómögulegt. Eftir það er verkefnið þannig séð komð. Þá förum við útí „security protocols“ eins og að ef bílinn fer útúr range-i frá fjarstýringunni, þá fer hann hægt og rólega til baka til þess að komast aftur í range í stað þess að hætta einfaldlega að virka.

Markhópurinn fyrir verkefnið yrði líklega fólk á öllum aldri sem hefur áhuga á að keyra bíl sem þau geta stjórnað með fjarstýringu, vilja sjá hvar bílinn er staðsettur og að geta séð það sem bílinn sér beint fram á við og nota arminn til að taka upp hluti og koma með þá.
Svona bílar geta haft ýmis notanagildi, eins og einföld skemmtun, reyna að njósna um einhvern eða til að gera tilraunir varðandi keyrslu, myndir og þess háttar. Svo geturu notað þá í myndatöku hér og þar og kannski til þess að njósna um annað fólk.
Okkur báðum langaði að búa til bíl sem við sjálfir getum forritað því að það er mjög gagnlegt að kunna það. Við munum læra mikið á þessu verkefni þar sem við munum gera eins mörg ´features´ í bílinn og við getum. T.d. security mode, sem myndi láta bílinn fara í þá átt sem hljóð kemur frá. Við getum sett speaker á bílinn og talað í gengum hann eða látið hann að búa til hljóð. Svo þarf að finna kerfi sem passar að bílinn sé öruggur og bruni ekki á vegg að fram af borði og skemmist.


\begin{figure}[h]
\includegraphics[scale=.3]{img/system}
\end{figure}
