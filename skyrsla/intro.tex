\section{Inngangur}
verk 1.
Arduino myndavel med gyroscope
Verkefni okkar er ad bua til arm med myndarvel sem mydi alltaf halda halla.  Vid aetlum ad nota motora sem myndu tengjast Arduino og forrita tha thannig ad their myndu hreida sig midad vid thad sem gyroscope segir. Vid reynum lika ad tha livestream ur myndarvelini og sja myndina I tolvunni a medan vid erum ad keyra forritid. Markmidid er ad tengja forritid vid bil sem vid settum sama a fyrsta onn og stjorna arminn I gegnum festringuna. Svona myndarvel getur verid notud i drone eda bilum med arm sem their gatu hreyft and haldid myndarvelinni steady med rettum halla. Robotin i heild mun bera svipadur og það sem við gerðum á 1 önn nema við bætum við nokkrum features í hann. 

verk 2.
Við ætlum að búa til drone sem notar síma með android stýrikerfninu til að stjórna flugi og stefnu, og ætlum að koma fyrir myndavél sem annaðhvort tekur upp myndband eða sendir live feed í síma.
Við ætlum að koma þessu í verk með því að koma Arduino stjórnborði fyrir á toppinum á drónanum sem verður með wifi module og radio signal module  og mögulega bluetooth module ef við finnum þannig, tengt við, sem okkur skilst að ætti að vera nóg til að stjórna honum. Eins og stendur erum við ekki alveg vissir hvaða tungumál við ætlum að notast við en þegar það kemur í ljós læt ég það hérna inn. 
Við fengum drone og tölvu frá kennaranum og ætlum að byrja verkefnið á því að tengja allt saman, stóra markmiðið er að fá live feed í síman og geta stjórnað dronin í gegnum síma. Við ætlum að nota tilbúið app frá Playstore sem fjarstýringu en forritum drónann sjálfir. Tengingin mun fara annaðhvort í gegnum gsm, bluetooth eða radio signal. Þegar við erum komnir með tengingu í síma ætlum við að reyna að fá live feed myndband, sem gæti verið flóknara, en ekki ómögulegt. Þá eru öll basics komin. Þá förum við útí „security protocols“ eins og að ef dróninn nær ekki sambandi við símann, þá fer hann hægt og rólega niður á jörðu í stað þess að hætta einfaldlega.
Markhópurinn fyrir verkefnið yrði líklega fólk á öllum aldri sem hefur áhuga á að fljúga dróna sem þau geta stjórnað með símanum sínum, þar sem flestallir eiga svoleiðis og vilja sjá hvar dróninn er staðsettur og að geta séð það sem dróninn sér beint fram á við. 
Svona drónar geta haft ýmis notanagildi, eins og einföld skemmtun, reyna að njósna um einhvern eða til að gera tilraunir varðandi flug, myndir og þess háttar. Svo geturu notað þá á myndatöku úr lofti eða einfaldlega til þess að sjá upp á hillur sem þú sérð ekki upp á.
Okkur báðum langaði að búa til drone sem við sjálfir getum forritað því að það er mjög gagnlegt að kunna það. Við munum læra mikið á þessu verkefni þar sem við munum gera eins mörg ´features´ í dronin og við getum. T.d security mode sem myndi láta dróninn fljúga í áttina að staðnum sem hljóð eða hreyfingar koma frá... Ef við fáum nógu langann tima þá getum við búið til hendi sem dróninn gæti notað til að lyfta hlutum upp. Við getum sett speaker á drónin og talað í gengum hann eða látið hann að búa til hljóð. Svo þarf að finna kerfi sem passar að dróninn sé „stable“ og hann hvolfi ekki sjálfum sér, og að hann bruni ekki á veggi og brotlendi í kjölfar. 
Ef allt fer á versta veg endum við líklega á því að nota fjarstýringu til að stýra drónanum og setjum live feed í símann í stað þess að streyma báðu hvoru í símann.

\begin{figure}[h]
\includegraphics[scale=.3]{img/system}
\end{figure}
